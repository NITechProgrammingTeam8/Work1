\documentclass[uplatex,12pt]{jsarticle}
\usepackage[dvipdfmx]{graphicx}
\usepackage{url}
\usepackage{listings,jlisting}
\usepackage{ascmac}
\usepackage{amsmath,amssymb}

%ここからソースコードの表示に関する設定
\lstset{
  basicstyle={\ttfamily},
  identifierstyle={\small},
  commentstyle={\smallitshape},
  keywordstyle={\small\bfseries},
  ndkeywordstyle={\small},
  stringstyle={\small\ttfamily},
  frame={tb},
  breaklines=true,
  columns=[l]{fullflexible},
  numbers=left,
  xrightmargin=0zw,
  xleftmargin=3zw,
  numberstyle={\scriptsize},
  stepnumber=1,
  numbersep=1zw,
  lineskip=-0.5ex
}
%ここまでソースコードの表示に関する設定

\title{知能プログラミング演習II 課題1}
\author{グループ8\\
  29114060 後藤 拓也\\
}
\date{2019年10月13日}

\begin{document}
\maketitle

\paragraph{提出物} rep1

\paragraph{グループ} グループ8

\paragraph{グループメンバー}
\begin{center}
\begin{tabular}{|c|c|c|}
  \hline
  学生番号&氏名&貢献度比率\\
  \hline\hline
  29114003&青山周平&NoData\\
  \hline
  29114060&後藤拓也&NoData\\
  \hline
  29114116&増田大輝&NoData\\
  \hline
  29114142&湯浅範子&NoData\\
  \hline
  29119016&小中祐希&NoData\\
  \hline
\end{tabular}
\end{center}
\paragraph{自分の役割} 最良優先探索とA*アルゴリズムのパラメータ調整
\\\\\\\\\\\\\\
%%%%%%%%%%%%%%%%%%%%%%%%%%%%%%%%%%%%%%%%%%%%%%%%%%%%%%%%%%%%%%%%%%%%%%%%%%%%%%
%%%%%%%%%%%%%%%%%%%%%%%%%%%%%%%%%%%%%%%%%%%%%%%%%%%%%%%%%%%%%%%%%%%%%%%%%%%%%%
\section{課題1}
%%%%%%%%%%%%%%%%%%%%%%%%%%%%%%%%%%%%%%%%%%%%%%%%%%%%%%%%%%%%%%%%%%%%%%%%%%%%%%
\subsection{最良優先探索}
最良推定法は各ノードにおけるヒューリスティック値をもとに探索を進めていく方法で,直前のノードから次のヒューリスティック値だけを見て,行き当たりばったりに探索を進める山登り法とは異なり,過去のデータ(これから訪れる可能性をもったノード)のヒューリスティック値をOpenListに保存するので,山登り法よりも最適な探索が可能である. 初期のパラメータを用いて探索を行うと下図1のような木構造を持つ探索となる.
\begin{figure}[htbp]
 \begin{center}
  \includegraphics[width = 10cm, pagebox = cropbox, clip]{最良優先探索ver1.pdf}
 \end{center}
 \caption[]{初期パラメータにおける木構造モデル}\label{fig:fig1.1}
\end{figure}

探索経路の解としては,
\begin{center}
[L.A.Airport→Hollywood→Pasadena→Las Vegas]
\end{center}
と4STEPで進めるはずが,探索ノード(親ノードの推移)としては,
\begin{center}
[L.A.Airport→Hollywood→DownTown→Sandiego→Pasadena→LasVegas]
\end{center}
と, 7STEPを踏んでいる. Hollywoodから次に進む際にPasadenaへ進むのではなく, 1度Downtownへ進んでいるのである. そのため,この改善策として, 「Pasadena:node[7]のヒューリスティック値:h(7)をDowntown:node[6]:h(6)よりも小さくする」方法を取る. そのパラメータ調整後の木構造が下図2のようになる.
\begin{figure}[htbp]
 \begin{center}
  \includegraphics[width = 10cm, pagebox = cropbox, clip]{最良優先探索ver2.pdf}
 \end{center}
 \caption[]{パラメータ調整後の木構造モデル}\label{fig:fig1.2}
\end{figure}
\\
探索ノード(親ノードの推移)として,
\begin{center}
[L.A.Airport→Hollywood→Pasadena→Las Vegas]
\end{center}
と最適パスとなった.\\
欠点としては, そのノードに固有なヒューリスティックな値を使って探索を進めていくので, すべてのノードのうちで最小のヒューリスティックの値をもつノードが無限に生成される場合は, 目標ノードに到達できない.\\


%%%%%%%%%%%%%%%%%%%%%%%%%%%%%%%%%%%%%%%%%%%%%%%%%%%%%%%%%%%%%%%%%%%%%%%%%%%%%%
\subsection{A*アルゴリズム}
A*アルゴリズムはそのノードまでのコストの合計とそのノードのヒューリステック値の和をとる. 最良優先探索の弱点でさえも初期ノードからのコストを考慮することで, 正解にはたどり着ける.\\

初期のパラメータを用いて探索を行うと下図3のような木構造を持つ探索となる.\\\\\\\\
\begin{figure}[htbp]
 \begin{center}
  \includegraphics[width = 10cm, pagebox = cropbox, clip]{Astar_ver1.pdf}
 \end{center}
 \caption[]{初期パラメータにおける木構造モデル}\label{fig:fig1.3}
\end{figure}

探索経路の解としては,
\begin{center}
[L.A.Airport→UCLA→Hollywood→Pasadena→Las Vegas]
\end{center}
と5STEPで進めるはずが, 探索ノード(親ノードの推移)としては,
\begin{center}
[L.A.Airport→Hollywood→UCLA→Hollywood→Pasadena→DisneyLand→Dawntown→LasVegas]
\end{center}
と, 8STEPを踏んでいる.余分な探索として, 2箇所が上げられる.
\begin{enumerate}
 \item 始めにHollywood:node[2]に進まずに, UCLA:node[1]に進んでほしい.
 \item 最後にPasadena:node[7]からDisneyLand:node[8]に進まず, LasVegas:node[8]に進んでほしい.
\end{enumerate}

よって, これらを改善するために以下の2つを変更してみる.
\begin{enumerate}
 \item UCLA:node[1]のヒューリスティック値:h(1)=7をHollywood:node[2]のヒューリスティック値:h(2)=4よりも小さくする. (既にコストgの値は小さいので.)
 \item Pasadena:node[7]からDisneyLand:node[8]のコストをPasadena:node[7]からDisneyLand:node[8]のコストよりも小さくする. (既にヒューリスティック値はh(9)よりも小さいので.)\\
その結果, 下図4のようになった.
\begin{figure}[htbp]
 \begin{center}
  \includegraphics[width = 10cm, pagebox = cropbox, clip]{Astar_ver2.pdf}
 \end{center}
 \caption[]{2つのパラメータ変更後の木構造モデル}\label{fig:fig1.3}
\end{figure}
\end{enumerate}
これによって,探索経路の解と同じ5STEPの探索ノード(親ノード)の推移を作ることができた.\\
なお, 始めの分岐でUCLA:node[1]を選ばず, Hollywood:node[2]を選ぶようにパラメータを調整することで, 4STEP探索も可能となる.(木構造は最良優先探索と同様になるので省略.)\\

%%%%%%%%%%%%%%%%%%%%%%%%%%%%%%%%%%%%%%%%%%%%%%%%%%%%%%%%%%%%%%%%%%%%%%%%%%%%%%
% 参考文献
\begin{thebibliography}{99}
\bibitem{notty} Javaによる知能プログラミング入門 --著:新谷 虎松 \\
\bibitem{notty} 知識システムの実装基礎 --著:新谷 虎松/大園 忠親/白松 俊 \\
\bibitem{notty} java CSV出力 --著:TECH Pin \\
\url{https://tech.pjin.jp/blog/2017/10/17/【java】csv出力のサンプルコード}
\bibitem{notty} java List配列処理と変換 --著:Samurai Blog \\
\url{https://www.sejuku.net/blog/16155}
\bibitem{notty} Gitでブランチを作成する方法 --著:ProEngineer \\
\url{https://proengineer.internous.co.jp/content.columnfeature/7633}
\bibitem{notty} Gitレポジトリの変更と取得 --著:GitHub ヘルプ\\
\url{https://help.github.com/ja/articles/getting-changes-from-a-remote-repository}
\bibitem{notty} LaTex 箇条書き --著:LaTexコマンド集\\
\url{http://www.latex-cmd.com/struct/list.html}
\end{thebibliography}

\end{document}